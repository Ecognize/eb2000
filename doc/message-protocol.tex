\documentclass[a4paper]{article}
\usepackage{xunicode}
\usepackage{fontspec}
\usepackage{xltxtra}
% \usepackage{fullpage}
\usepackage[left=30mm,right=15mm,top=20mm,bottom=20mm,includefoot]{geometry}
\usepackage[small,compact]{titlesec}
\usepackage{multicol}
\usepackage{wrapfig}
\usepackage{indentfirst}
% \usepackage{float}
\usepackage{amsmath}
\usepackage{setspace}
\XeTeXinputencoding utf8
% \usepackage[pdft]{graphicx}
%\usepackage[utf8]{inputenc}
\usepackage[ukrainian]{polyglossia}
\title{Протокол сообщений Electro Body 2000}
\author{Олексій Протас}
\usepackage[pdfauthor={Олексій Протас},pdftitle={Протокол сообщений Electro Body 2000},colorlinks]{hyperref}
\providecommand{\term}[2]{\emph{#1 (#2)}}
\providecommand{\article}[1]{\emph{#1~et~al.}}
\renewcommand{\subsubsection}[1]{\addcontentsline{toc}{subsubsection}{\arabic{section}.\arabic{subsection}.\arabic{subsubsection}.~#1}\stepcounter{subsubsection}\textbf{\arabic{section}.\arabic{subsection}.\arabic{subsubsection}.~#1}}
\setmainfont[Mapping=tex-text]{Linux Libertine O}
\setmonofont[Mapping=tex-text]{FreeMono}
\makeatletter
% \bibliographystyle{unsrt}
\renewcommand{\@biblabel}[1]{#1.}
\newenvironment{itemize*}{\begin{itemize}\setlength{\parsep}{0pt}\setlength{\parskip}{0pt}\setlength{\itemsep}{-14pt}}{\end{itemize}}
\makeatother
\onehalfspacing
\begin{document}
    \addfontfeature{Numbers=OldStyle}
    \maketitle
    \tableofcontents
    \pagebreak
    \section{Общие положения}
    
    Командные сообщения передаются в двоичном формате. Описательные — в XML.
    
    \section{Командный («игровой») протокол}
    
    \subsection{Схема протокола}
    
    \subsection{Запросы клиента}
    
    \subsubsection{rControlPress}
    
    Извещает сервер о том, что игрок нажал клавишу управления персонажем. Аргумент — направление из множества
\{\textit{Влево},\textit{Вправо},\textit{Прыжок},\textit{Влево-прыжок},\textit{Вправо-прыжок}\}. При последователном
посыле противоречивых сообщений о направлении (сначала влево, потом вправо), клиент может опустить соответствующий
вызов команды \texttt{rControlRelease}. При посыле нажатия \textit{Вверх} при зажатой клавише направления аналогично
посылу соответсвующего прыжка с направлением.
    
    \subsubsection{rControlRelease}
    
    Извещает сервер о прекращении нажатия кнопки управления. Стоит заметить, что механика игры предусматривает
«шаговую» модель при горизонтальном передвижении, потому эффективный конец движения может отдаляться во времени от
посланого клиентом сообщения. Аргументы совпадают с аргументами \texttt{rCon\-trol\-Press}.
    
    \subsubsection{rSpecialAbility}
    
    Извещает сервер об активации игроком специальной возможности (телепортация, выстрел, открытие двери \emph{et
cetera}). Операция атомарна, для посыла нескольких выстрелов необходимо отправить несколько команд, в отличие от
предыдущих схем «нажатия-отжатия».
    
    \subsubsection{rPauseRequest}
    
    Попытка остановить или продолжить выполнение игры на сервере и всех клиентах. Сервер обрабатывает этот запрос в
соответствии со своей политикой и, в случае разрешения, посылает всем клиентам уведомление о паузе \texttt{aPauseGame},
а запрашивающему клиенту дополнительное извещение об успехе или неудаче запроса \texttt{aPauseResponce}. 
    
    \subsubsection{rChatMessage}
    
    Посылает текстовое сообщение на сервер для оглашения подключенным игрокам. Сообщение передается в кодировке UTF-8.
    
    \subsubsection{rLeaveGame}
   
    Извещает сервер об выходе клиента из игры. Используется для корректного отключения, при этом персонаж «умирает», в
отличие от обрыва связи, когда персонаж переходит в ghost-режим.
   
    \subsubsection{rWaitForSync}
    
    Извещает сервер о неспособности своевременно провести рассчеты. При разрешении сервера при этом у всех иных игроков
происходит вынужденная неперебиваемая пауза, а клиент, инициировавший паузу, получает от сервера явное обновление
ситуации игрового экрана.
   
    \subsection{Извещения сервера}
    
    \section{Протокол описаний и загрузки ресурсов}
    
    TODO
    
\end{document}
