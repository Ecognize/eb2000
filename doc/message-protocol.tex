\documentclass[a4paper]{article}
\usepackage{xunicode}
\usepackage{fontspec}
\usepackage{xltxtra}
% \usepackage{fullpage}
\usepackage[left=30mm,right=15mm,top=20mm,bottom=20mm,includefoot]{geometry}
\usepackage[small,compact]{titlesec}
\usepackage{multicol}
\usepackage{wrapfig}
\usepackage{indentfirst}
% \usepackage{float}
\usepackage{amsmath}
\usepackage{setspace}
\XeTeXinputencoding utf8
% \usepackage[pdft]{graphicx}
%\usepackage[utf8]{inputenc}
\usepackage[ukrainian]{polyglossia}
\title{Протокол сообщений Electro Body 2000}
\author{Олексій Протас}
\usepackage[pdfauthor={Олексій Протас},pdftitle={Протокол сообщений Electro Body 2000},colorlinks]{hyperref}
\providecommand{\term}[2]{\emph{#1 (#2)}}
\providecommand{\article}[1]{\emph{#1~et~al.}}
\renewcommand{\subsubsection}[1]{\addcontentsline{toc}{subsubsection}{\arabic{section}.\arabic{subsection}.\arabic{subsubsection}.~#1}\stepcounter{subsubsection}\textbf{\arabic{section}.\arabic{subsection}.\arabic{subsubsection}.~#1}}
\setmainfont[Mapping=tex-text]{Linux Libertine O}
\setmonofont[Mapping=tex-text]{FreeMono}
\makeatletter
% \bibliographystyle{unsrt}
\renewcommand{\@biblabel}[1]{#1.}
\newenvironment{itemize*}{\begin{itemize}\setlength{\parsep}{0pt}\setlength{\parskip}{0pt}\setlength{\itemsep}{-14pt}}{\end{itemize}}
\makeatother
\onehalfspacing
\begin{document}
    \addfontfeature{Numbers=OldStyle}
    \maketitle
    \tableofcontents
    \pagebreak
    \section{Общие положения}
    
    Командные сообщения передаются в двоичном формате. Описательные — в XML.
    
    \section{Командный («игровой») протокол}
    
    \subsection{Схема протокола}
    
    \subsection{Запросы клиента}
    
    \subsubsection{rControlPress}
    
    Извещает сервер о том, что игрок нажал клавишу управления персонажем. Аргумент — направление из множества
\{\textit{Влево},\textit{Вправо},\textit{Прыжок},\textit{Влево-прыжок},\textit{Вправо-прыжок}\}. При последователном
посыле противоречивых сообщений о направлении (сначала влево, потом вправо), клиент может опустить соответствующий
вызов команды \texttt{rControlRelease}. При посыле нажатия \textit{Вверх} при зажатой клавише направления аналогично
посылу соответсвующего прыжка с направлением.
    
    \subsubsection{rControlRelease}
    
    Извещает сервер о прекращении нажатия кнопки управления. Стоит заметить, что механика игры предусматривает
«шаговую» модель при горизонтальном передвижении, потому эффективный конец движения может отдаляться во времени от
посланого клиентом сообщения. Аргументы совпадают с аргументами \texttt{rCon\-trol\-Press}.
    
    \subsubsection{rSpecialAbility}
    
    Извещает сервер об активации игроком специальной возможности (телепортация, выстрел, открытие двери \emph{et
cetera}). Операция атомарна, для посыла нескольких выстрелов необходимо отправить несколько команд, в отличие от
предыдущих схем «нажатия-отжатия».
    
    \subsubsection{rPauseRequest}
    
    Попытка остановить или продолжить выполнение игры на сервере и всех клиентах. Сервер обрабатывает этот запрос в
соответствии со своей политикой и, в случае разрешения, посылает всем клиентам уведомление о паузе \texttt{aPauseGame},
а запрашивающему клиенту дополнительное извещение об успехе или неудаче запроса \texttt{aPauseResponce}. 
    
    \subsubsection{rChatMessage}
    
    Посылает текстовое сообщение на сервер для оглашения подключенным игрокам. Сообщение передается в кодировке UTF-8.
    
    \subsubsection{rLeaveGame}
   
    Извещает сервер об выходе клиента из игры. Используется для корректного отключения, при этом персонаж «умирает», в
отличие от обрыва связи, когда персонаж переходит в ghost-режим.
   
    \subsubsection{rWaitForSync}
    
    Извещает сервер о неспособности своевременно провести рассчеты либо при возвращении из разрыва связи. При разрешении
сервера при этом у всех иных игроков происходит вынужденная неперебиваемая пауза фиксированной длительности, а клиент,
инициировавший паузу получает от сервера явное обновление ситуации игрового экрана.
   
    \subsection{Извещения сервера}
    
    \subsubsection{aCharacterLockStatus}
    
    Извещает игрока, о том, что его персонаж потерял управление. Потеря управления происходит, например, в полете, или
при анимации телепорта. Основной смысл сообщения предназначен для бота при анализе возможных ходов в данный момент
времени. Аргументы — состояние \{\textit{Блокирован},\textit{Разблокирован}\}, опционально — список имен команд,
доступных в данном состоянии (стрельба в полете возможна, а при телепортации — нет).
    
    \subsubsection{aSpotObject}
    
    Извещает о появлении на текущем экране обьекта определенного класса. Аргументы — положение, ориентация, имя класса,
внутренняя метка. При переходе в новую комнату это сообщение может просылаться для уже существующих обьектов, чтобы
перезадать их положение. Так же, при разсинхронизации клиента, сервер во время вынужденной паузы, пересылает клиенту
заново все положения на текущем экране.
    
    \subsubsection{aKillObject}
    
    Извещает клиента об устранении обьекта. Аргумент — метка объекта, способ устранения из набора
\{\textit{Устранить},\textit{Убить}\}, в первом случае, обьект просто исчезает, во втором — исчезает с характерной
анимацией или звуком (курочка, патроны итд). При смерти пушек, обьект типа «пушка» погибает со взрывом, а объект типа
«останки пушки» появляется на том же месте. Сходным образом работают батарейки.
    
    \subsubsection{aMotionChange}
    
    Изменяет вектор движения объекта. Аргумент — метка объекта, новый вектор движения или специальное значение,
оглашающее об остановке движения. NB: Куся, как происходит падение и прыжки — просто или с ускорениями?
    
    \subsubsection{aSpecialMove}
    
    Сообщает о том, что объект совершает действие, требующее визуального или звукового отображения. Аргумент — метка
объекта и имя класса действия.
    
    \subsubsection{aChangeRoom}
    
    Сообщает игроку про его перемещение в другую комнату. Срабатывет при пересечении границ экрана, телепорте, входе в
двери, восстановлении после смерти. Аргументы — номер новой комнаты, положение и ориентация игрока. При входе в игру
игрок получает первое сообщение этого класса, которое характеризует его начальную точку и начальный момент игрового
времени для синхронизации часов.
    
    \subsubsection{aPauseGame}
    
    \subsubsection{aPauseResponce}
    
    \subsubsection{aChatMessage}
    
    \section{Протокол описаний и загрузки ресурсов}
    
    TODO
    
\end{document}
